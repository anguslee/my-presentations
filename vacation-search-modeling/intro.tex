%
% intro.tex
% 度假搜索业务模型简介
%

\documentclass{report}

\usepackage{CJKutf8}
\usepackage[T1]{fontenc}
\usepackage{mathtools}
\usepackage{indentfirst}

\everymath{\displaystyle}
\pagestyle{headings}
\pagenumbering{roman}

\begin{document}
\begin{CJK}{UTF8}{gbsn}

\title{度假搜索业务介绍}
\author{李庚}
\date{\today}

\bibliographystyle{plain}

\maketitle
\tableofcontents
\newpage

\chapter{度假部门业务概述}

卖货:尽可能多的将供应商提供的旅游度假产品卖给使用网站的用户。部门的主要业绩衡量标准包括:

\begin{itemize}
  \item { GMV: Gross Merchandise Value,即交易额 }
  \item { 毛利:$GMV - \text{成本} $ }
\end{itemize}


GMV计算公式:

\begin{equation}
  GMV = \sum_{i \in S}{UV_i \times A_i \times R_i \times M_i}
\end{equation}

\begin{itemize}
  \item { S: 访问渠道 }
  \item { UV: 独立用户访问量 }
  \item { A: 服务可用率 }
  \item { R: 订单转化率 }
  \item { M: 平均用户交易额 }
\end{itemize}

服务可用率:
\begin{equation}
  A_i = \frac{\text{满足性能要求的正确返回请求数}_i}{\text{全部请求数}_i}
\end{equation}

订单转化率:
\begin{equation}
  R_i = \frac{\text{订单量}_i}{\text{列表页访问量}_i} = \frac{\text{详情页访问量}_i}{\text{列表页访问量}_i} \times \frac{\text{订单量}_i}{\text{详情页访问量}_i}
\end{equation}

每个项目的目标一定要围绕提升上述某个指标,或者某几项指标的乘积。

\chapter{度假搜索在部门业务中的作用}

帮助用户快速找到需要的产品。主要目标是提升我们度假频道的这几个指标:$ UV, A, R $

\section{度假搜索小组职能划分}



需求类型对应的目标举例:

\begin{itemize}
  \item { SEO,流量运营: 提升 $ UV $ }
  \item { 展示优化、相关产品推荐、信息抽取:提升 $ R $ }
  \item { 后台性能优化、架构调整优化:提升 $ A $ }
\end{itemize}

\chapter{技术原则}

\section{容错设计}

外部故障是\emph{不可避免}的,且发生具有随机性。比如:硬件损坏、能源失效、网络连接中断等。我们在系统设计上能做的是,控制这些意外情况的发生对我们系统功能的影响范围和程度。即尽量保证:\emph{A}

典型策略:

\begin{itemize}
  \item { 杜绝单点依赖 }
  \item { 对外部访问要有恰当的异常处理 }
\end{itemize}

\section{测量效果}

无论\emph{UV, A, R, M}都要有合理的测量标准。\emph{UV},\emph{R}和\emph{M}的标准比较直观,\emph{A}的定义充分体现了两个方面的重要性:正确与性能。返回不正确结果与处理性能过差的请求都是影响可用性的。

为了进行效果测量进行的工作:

\begin{itemize}
  \item { 抽取评测 }
  \item { 转化率监控 }
  \item { 报警监控 }
\end{itemize}

\section{快速试错}

如果一项功能改动是否有用是无法预知的,必须发布到线上才能收到效果,那么为了避免错误的修改对线上产生过大的影响,我们需要尽快确认改动造成的影响是正向还是负向。

采用方法:灰度发布 + AB对比测试,对比的指标是两个环境的单用户贡献\emph{GMV}值。

\bibliography{math}

\clearpage
\end{CJK}
\end{document}