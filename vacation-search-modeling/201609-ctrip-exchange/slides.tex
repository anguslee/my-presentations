\documentclass[handout]{beamer}
% Replace the \documentclass declaration above
% with the following two lines to typeset your 
% lecture notes as a handout:
%\documentclass[handout]{beamer}
\usepackage{CJKutf8}
\usepackage[T1]{fontenc}
%\usepackage[utf8x]{inputenc}
\usepackage{graphicx}
\usepackage{subfigure}
\usepackage{mathtools}
\usepackage{ulem}
\usepackage{url}
\usepackage{pifont}
\usepackage{pgfplots}
\usepackage{textcomp}

\usepgfplotslibrary{external}
\pgfplotsset{width=10cm,compat=1.9}

%% \usepackage{tikz}
%% \usepackage{verbatim}
%% \usepackage[active,tightpage]{preview}
%% \PreviewEnvironment{center}
%% \setlength\PreviewBorder{10pt}

\usetikzlibrary{shapes,arrows}

\tikzexternalize
\everymath{\displaystyle}

% There are many different themes available for Beamer. A comprehensive
% list with examples is given here:
% http://deic.uab.es/~iblanes/beamer_gallery/index_by_theme.html
% You can uncomment the themes below if you would like to use a different
% one:
%\usetheme{AnnArbor}
%\usetheme{Antibes}
%\usetheme{Bergen}
%\usetheme{Berkeley}
%\usetheme{Berlin}
%\usetheme{Boadilla}
%\usetheme{boxes}
%\usetheme{CambridgeUS}
%\usetheme{Copenhagen}
%\usetheme{Darmstadt}
%\usetheme{default}
%\usetheme{Frankfurt}
%\usetheme{Goettingen}
%\usetheme{Hannover}
%\usetheme{Ilmenau}
%\usetheme{JuanLesPins}
%\usetheme{Luebeck}
%\usetheme{Madrid}
%\usetheme{Malmoe}
%\usetheme{Marburg}
%\usetheme{Montpellier}
%\usetheme{PaloAlto}
%\usetheme{Pittsburgh}
%\usetheme{Rochester}
%\usetheme{Singapore}
%\usetheme{Szeged}
\usetheme{Warsaw}

%% \newcommand{chartnumproducts} {
%% }



\begin{document}

\begin{CJK}{UTF8}{gbsn}

\title{度假搜索排序与系统监控简介}

% A subtitle is optional and this may be deleted
% \subtitle{}

\author{李庚\inst{1}}
% - Give the names in the same order as the appear in the paper.
% - Use the \inst{?} command only if the authors have different
%   affiliation.

\institute[Qunar.com] % (optional, but mostly needed)
{
  \inst{1}
  旅游度假事业部-搜索及频道 \\
  Qunar.com
}
% - Use the \inst command only if there are several affiliations.
% - Keep it simple, no one is interested in your street address.

\date{\today}

\AtBeginSection[]
{
  \begin{frame}<beamer>{纲要}
    \tableofcontents[currentsection,currentsubsection]
  \end{frame}
}

\AtBeginSubsection[]
{
  \begin{frame}<beamer>{纲要}
    \tableofcontents[currentsection,currentsubsection]
  \end{frame}
}


\begin{frame}
  \titlepage
\end{frame}

%% \begin{frame}{纲要}
%%   \tableofcontents
%% \end{frame}

\section{搜索排序基本原理简介}

\begin{frame}{参与排序主要因素}
  \begin{itemize}
    \item {行为反馈}
    \item {产品质量}
    \item {供应商评级}
  \end{itemize}
  \begin{block}<2->{对于产品r,其在查询词q上的排序计算公式为:}
    $$ score_{q,r} = a \times {\color{blue}{\text{行为反馈}_{q,r}}} + b \times {\color{blue}{\text{产品质量}}} + c \times {\color{blue}{\text{供应商评级}}} $$
  \end{block}
\end{frame}

\begin{frame}{用户行为反馈}
  \begin{itemize}
  \item {奖励转化率高产品:L-D, D-P等因素}
  \item {惩罚有了展示之后无转化产品}
  \end{itemize}
\end{frame}

\begin{frame}{产品本身质量}
  \begin{block}{原则:用户关注的出行信息是否完整}
    \begin{itemize}
    \item {费用相关:票是否含税;是否有低价钓鱼行为等}
    \item {产品图片:不能缺失}
    \item {信息准确,自身不能前后矛盾}
    \end{itemize}
  \end{block}
  
\end{frame}

\begin{frame}{供应商评级}
  \begin{itemize}
  \item {投诉率}
  \item {电话接通率,在线客服应答率}
  \item {支付率,销售金额}
  \item {新供应商或者新产品会有一定的加分}
  \end {itemize}
  \begin{block}<2->{供应商QChat客服沟通监控点}
    \begin{itemize}
    \item {直接统计:客服的回复率,尤其是\emph{及时回复率}}
    \item {语义分析:通过分析会话session中的关键词发现疑似投诉、用户的主要需求点等}
    \end{itemize}
  \end{block}
\end{frame}


\section{系统监控的原则简介}

\begin{frame}{为什么要做监控?}
  % \frametitle<presentation>{推荐阅读}
  我们维护的技术系统复不复杂?一旦失效了怎么办?
  \begin{thebibliography}{10}
    
  \beamertemplatearticlebibitems
  % Followed by interesting articles. Keep the list short. 

  \bibitem{hcsf}
    Richard I. Cook
    \newblock {How Complex Systems Fail}
    \newblock \url{http://web.mit.edu/2.75/resources/random/How\%20Complex\%20Systems\%20Fail.pdf}

  \end{thebibliography}

  \beamertemplate

  \begin{block}<2->{\emph {灾难性失效}的特性}
    \begin{enumerate}
    \item<2> {\emph{可靠性}很难度量,只能度量\emph{不可靠}。}
    \item<3> {失效的\emph{风险一定存在},尤其对于快速迭代演变的系统。}
    \item<4-> {失效的原因一般十分复杂,且很少具有可重复性,也\emph{没有单一的root cause.}}
    \item<5-> {亡羊补牢,为时已晚。即事后的补救措施对将来的意义不一定很大。}
    \end{enumerate}
  \end{block}

\end{frame}

\begin{frame}{监控的目的:及早发现异常,避免\emph {灾难性}的失效}
  \begin{block}{思路与方法}
    \begin{itemize}
    \item {合理的技术架构设计}
    \item {重视指标的度量}
    \end{itemize}
  \end{block}
\end{frame}

\begin{frame}{合理的技术架构设计是前提}
  \begin{enumerate}
  \item<1-> {保证最普遍的应用场景在需求演化过程中不失效。}
  \item<2-> {不断的消除性能瓶颈,必要时大胆重构。成功重构的代价远小于不断的打补丁。}
  \item<3-> {监控不宜过度与业务场景结合,业务需求的正确性由开发上线的流程保证而非通过监控保证。}
  \end{enumerate}
\end{frame}

\begin{frame}{重视指标的度量是基础}
  \begin{enumerate}
  \item<1-> {技术层面:\emph{分服务}统计响应时间,可用性,硬件报警状况,操作系统负载等。}
  \item<2-> {业务层面:\emph{分渠道}统计转化率,流量,交易额等。}
  \item<3-> {不要忽视度量指标的任何一次\emph{突变}。}
  \end{enumerate}
\end{frame}

\end{CJK}

\end{document}
