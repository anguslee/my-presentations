\documentclass{beamer}
% Replace the \documentclass declaration above
% with the following two lines to typeset your 
% lecture notes as a handout:
%\documentclass[handout]{beamer}
\usepackage{CJKutf8}
\usepackage[T1]{fontenc}
%\usepackage[utf8x]{inputenc}
\usepackage{graphicx}
\usepackage{subfigure}
\usepackage{mathtools}
\usepackage{ulem}
\usepackage{url}
\usepackage{pifont}
\usepackage{pgfplots}
\usepackage{textcomp}

\usepgfplotslibrary{external}
\pgfplotsset{width=10cm,compat=1.9}

%% \usepackage{tikz}
%% \usepackage{verbatim}
%% \usepackage[active,tightpage]{preview}
%% \PreviewEnvironment{center}
%% \setlength\PreviewBorder{10pt}

\usetikzlibrary{shapes,arrows}

\tikzexternalize
\everymath{\displaystyle}

% There are many different themes available for Beamer. A comprehensive
% list with examples is given here:
% http://deic.uab.es/~iblanes/beamer_gallery/index_by_theme.html
% You can uncomment the themes below if you would like to use a different
% one:
%\usetheme{AnnArbor}
%\usetheme{Antibes}
%\usetheme{Bergen}
%\usetheme{Berkeley}
%\usetheme{Berlin}
%\usetheme{Boadilla}
%\usetheme{boxes}
%\usetheme{CambridgeUS}
%\usetheme{Copenhagen}
%\usetheme{Darmstadt}
%\usetheme{default}
%\usetheme{Frankfurt}
%\usetheme{Goettingen}
%\usetheme{Hannover}
%\usetheme{Ilmenau}
%\usetheme{JuanLesPins}
%\usetheme{Luebeck}
%\usetheme{Madrid}
%\usetheme{Malmoe}
%\usetheme{Marburg}
%\usetheme{Montpellier}
%\usetheme{PaloAlto}
%\usetheme{Pittsburgh}
%\usetheme{Rochester}
%\usetheme{Singapore}
%\usetheme{Szeged}
\usetheme{Warsaw}

%% \newcommand{chartnumproducts} {
%% }



\begin{document}

\begin{CJK}{UTF8}{gbsn}

\title{度假搜索排序与系统监控简介}

% A subtitle is optional and this may be deleted
% \subtitle{}

\author{李庚\inst{1}}
% - Give the names in the same order as the appear in the paper.
% - Use the \inst{?} command only if the authors have different
%   affiliation.

\institute[Qunar.com] % (optional, but mostly needed)
{
  \inst{1}
  旅游度假事业部-搜索及频道 \\
  Qunar.com
}
% - Use the \inst command only if there are several affiliations.
% - Keep it simple, no one is interested in your street address.

\date{\today}

\AtBeginSection[]
{
  \begin{frame}<beamer>{纲要}
    \tableofcontents[currentsection,currentsubsection]
  \end{frame}
}

\AtBeginSubsection[]
{
  \begin{frame}<beamer>{纲要}
    \tableofcontents[currentsection,currentsubsection]
  \end{frame}
}


\begin{frame}
  \titlepage
\end{frame}

\begin{frame}{纲要}
  \tableofcontents
\end{frame}

\section{搜索排序基本原理简介}

\begin{frame}{参与排序主要因素}
  \begin{itemize}
    \item {行为反馈}
    \item {产品质量}
    \item {供应商评级}
  \end{itemize}
  \begin{block}{对于产品r,其在查询词q上的排序计算公式为:}
    $$ score_{q,r} = a \times \text{行为反馈}_{q,r} + b \times \text{产品质量} + c \times \text{供应商评级} $$
  \end{block}
\end{frame}

\begin{frame}{用户行为反馈}
  \begin{itemize}
  \item {奖励转化率高产品:L-D, D-P等因素}
  \item {惩罚有了展示之后无转化产品}
  \end{itemize}
\end{frame}

\begin{frame}{产品本身质量}
  \begin{block}{原则:用户关注的出行信息是否完整}
    \begin{itemize}
    \item {费用相关:票是否含税;是否有低价钓鱼行为等}
    \item {产品图片:不能缺失}
    \item {信息准确,自身不能前后矛盾}
    \end{itemize}
  \end{block}
  
\end{frame}

\begin{frame}{供应商评级}
  \begin{itemize}
  \item {投诉率}
  \item {电话接通率,在线客服应答率}
  \item {支付率,销售金额}
  \item {新供应商或者新产品会有一定的加分}
  \end {itemize}
\end{frame}

\section{系统监控的原则简介}

\begin{frame}{监控的目的:及早发现异常,避免灾难性的失效}
  \begin{itemize}
  \item {对于不同系统,“失效”的标准不同}
  \item {失效的原因很可能十分复杂,一般不可重复。也没有单一的root cause}
  \item {亡羊补牢,为时已晚}
  \end{itemize}
\end{frame}

\begin{frame}{从失效中学习:合理的设计是前提}
  \begin{itemize}
  \item {保证最普遍的应用场景在需求演化过程中不失效。}
  \item {不断的消除性能瓶颈,必要时大胆重构。成功重构的代价远小于不断的打补丁。}
  \item {不宜过度与业务场景结合,开发监控程序本身的成本不可忽视。}
  \end{itemize}
\end{frame}

\begin{frame}{原则:变监控为度量}
  \begin{itemize}
  \item {技术层面:服务响应时间,可用性,硬件报警状况,操作系统负载等}
  \item {业务层面:转化率,流量,交易额等}
  \item {不要忽视被度量指标的任何一次突变}
  \end{itemize}
\end{frame}

\end{CJK}

\end{document}
