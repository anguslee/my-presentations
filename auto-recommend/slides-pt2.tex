\documentclass{beamer}
% Replace the \documentclass declaration above
% with the following two lines to typeset your 
% lecture notes as a handout:
%\documentclass[handout]{beamer}
\usepackage{CJKutf8}
\usepackage[T1]{fontenc}
%\usepackage[utf8x]{inputenc}
\usepackage{graphicx}
\usepackage{subfigure}
\usepackage{mathtools}
\usepackage{ulem}
\usepackage{url}
\usepackage{pifont}
\usepackage{pgfplots}
\usepackage{textcomp}

\usepgfplotslibrary{external}
\pgfplotsset{width=10cm,compat=1.9}

%% \usepackage{tikz}
%% \usepackage{verbatim}
%% \usepackage[active,tightpage]{preview}
%% \PreviewEnvironment{center}
%% \setlength\PreviewBorder{10pt}

\usetikzlibrary{shapes,arrows}

\tikzexternalize
\everymath{\displaystyle}

% There are many different themes available for Beamer. A comprehensive
% list with examples is given here:
% http://deic.uab.es/~iblanes/beamer_gallery/index_by_theme.html
% You can uncomment the themes below if you would like to use a different
% one:
%\usetheme{AnnArbor}
%\usetheme{Antibes}
%\usetheme{Bergen}
%\usetheme{Berkeley}
%\usetheme{Berlin}
%\usetheme{Boadilla}
%\usetheme{boxes}
%\usetheme{CambridgeUS}
%\usetheme{Copenhagen}
%\usetheme{Darmstadt}
%\usetheme{default}
%\usetheme{Frankfurt}
%\usetheme{Goettingen}
%\usetheme{Hannover}
%\usetheme{Ilmenau}
%\usetheme{JuanLesPins}
%\usetheme{Luebeck}
%\usetheme{Madrid}
%\usetheme{Malmoe}
%\usetheme{Marburg}
%\usetheme{Montpellier}
%\usetheme{PaloAlto}
%\usetheme{Pittsburgh}
%\usetheme{Rochester}
%\usetheme{Singapore}
%\usetheme{Szeged}
\usetheme{Warsaw}

%% \newcommand{chartnumproducts} {
%% }



\begin{document}

\begin{CJK}{UTF8}{gbsn}

\title{度假搜索的自动化推荐}

% A subtitle is optional and this may be deleted
\subtitle{实践与初步效果}

\author{李庚\inst{1}}
% - Give the names in the same order as the appear in the paper.
% - Use the \inst{?} command only if the authors have different
%   affiliation.

\institute[Qunar.com] % (optional, but mostly needed)
{
  \inst{1}
  旅游度假事业部-搜索及频道 \\
  Qunar.com
}
% - Use the \inst command only if there are several affiliations.
% - Keep it simple, no one is interested in your street address.

\date{\today}

\AtBeginSection[]
{
  \begin{frame}<beamer>{纲要}
    \tableofcontents[currentsection,currentsubsection]
  \end{frame}
}

\AtBeginSubsection[]
{
  \begin{frame}<beamer>{纲要}
    \tableofcontents[currentsection,currentsubsection]
  \end{frame}
}


\begin{frame}
  \titlepage
\end{frame}

%% \begin{frame}{纲要}
%%   \tableofcontents
%% \end{frame}

\section{技术原理回顾}

\begin{frame}{推荐方法质量评价标准}
  \begin{itemize}
    \item {成本是否合理}
    \item {\emph{能够带来多少实际的用户到订单转化}}
  \end{itemize}
\end{frame}

\begin{frame}{合理的推荐建立在什么基础上?}
  \begin{itemize}
  \item {知己:明确自己的优质内容的范围}
  \item {知彼:了解用户的需要}
  \item {推荐方法:将符合用户需要的优质内容给用户看}
  \end{itemize}
\end{frame}

\begin{frame}{两种常用手段}
  \begin{itemize}
  \item {
    协同过滤(Collaborate Filtering)
    \begin{enumerate}
    \item {对于无行为或者行为少的用户无能为力;}
    \item {当前统计到的影响用户比例:8\%}
    \item {
      带来的duv-p转化
      \begin{itemize}
      \item {度假频道11/17-11/23: 0.92\%}
      \item {同期列表页协同过滤推荐:[0.82\%-1.06\%]}
      \end{itemize}
    }
    \end{enumerate}
  }
  \item {
    特征匹配推荐(Contented Based)
  }
  \end{itemize}
\end{frame}

\begin{frame}{基于特征匹配的推荐}

  $$ A \rightarrow x, B \rightarrow x $$
  $$ A \rightarrow y, C \rightarrow z $$
  $$ B \Leftarrow {\color{blue}{y}},z ? $$

  \begin{itemize}
  \item { A和B具有共同偏好特征$ u_1 $ }
  \item { 产品x具有特征 $ p_1 $ }
  \item { 具有 $ u_1 $ 特征的人偏爱具备 $ p_1 $ 特征的产品 }
  \item { 产品y也具有特征 $ p_1 $ }
  \end{itemize}

\end{frame}

\begin{frame}{特征匹配度计算方法}
  \begin{itemize}
    \item { $ \text{令}U=\text{度假用户集合}, P=\text{度假产品集合}, $ }
    \item {
      通过行为反馈计算用户u对产品p的偏好程度
      \begin{itemize}
        \item {
          $ W_{u,p}, (u \in U, p \in P) $
        }
      \end{itemize}
    }
    \item {
      将产品和用户表示成他们的特征:
      \begin{itemize}
      \item { $ u=(u_1, u_2, \dots , u_m), u \in U $ }
      \item { $ p=(p_1, p_2, \dots , p_n), p \in P $ }
      \end{itemize}
    }
    \item {
      特征之间的相关性:
      \begin{itemize}
        \item { $ C_{u_x,p_y} = \sum_{u \in U, p \in P}{\frac{W_{u,p}}{|u| \times |p|}}$ }
      \end{itemize}
    } 
    
  \end{itemize}
\end{frame}



\section{实现与评估方法}

\begin{frame}{个性化排序实现概览}
  \begin{itemize}
  \item {依靠经验进行特征精简}
  \item {特征相关性模型计算,评估验证}
  \item {依据特征相关性模型和用户特征,对搜索结果逐步进行权重调整。}
  \end{itemize}
\end{frame}

\begin{frame}{用户特征$u_i$选择}
  \begin{itemize}
  \item { 出行方式偏好跟团or自由行 }
  \item { 酒店高星级偏好 }
  \item { 靠近热门商圈偏好 }
  \item { 靠近地铁线路偏好 }
  \item { 靠近火车站或机场偏好 }
  \item { 靠近观光景区偏好 }
  \item { 优惠产品(错峰,特价产品等)倾向 }
  \item { 带儿童出行倾向 }
  \end{itemize}
\end{frame}

\begin{frame}{产品特征$p_i$选择}
  \begin{itemize}
  \item { 商品有优惠(特卖特价、甩位等) }
  \item { 包含核心知名景点 }
  \item { 住高级酒店 }
  \item { 住宿靠近热门商圈 }
  \item { 住宿靠近地铁线路 }
  \item { 住宿靠近火车站或机场 }
  \item { 住宿靠近观光景区 }
  \item { 含精品免税店, 待整理 }
  \item { 含自由活动 }
  \item { 含节假日团期 }
  \item { 含错峰出行团期 }
  \item { 有儿童报价 }
  \end{itemize}
\end{frame}

\begin{frame}{特征相关性模型计算主要结论}
  \begin{enumerate}
  \item {含节假日团期、提供优惠、提供儿童价:无区分度的产品特征}
  \item {多数用户属性与自由行产品的关联度大,“偏好优惠”与“带儿童出行”属性除外}
  \end{enumerate}
\end{frame}

\begin{frame}{作用于搜索排序并评估结果}
  \begin{itemize}
  \item {
    同时间段上用小流量测试进行对比:排除所有可能的干扰因素
    \begin{itemize}
    \item {A: 置顶+个性化排序 }
    \item {B: 个性化排序}
    \item {C: 置顶}
    \end{itemize}
  }
  \item {
    量化个性化排序对搜索列表的影响:看不同环境中被个性化加权产品的曝光率有多大差别
  }
  \item {测试环境校验:确认测试系统本身不会带来转化率误差}
  \item {
    收集1/26-2/02的app端数据进行评估,主要看三个环境下uv维度的L-D, D-O,以及受个性化排序影响的用户占比
  }
  \end{itemize}
  
\end{frame}

\section{数据结论}

\begin{frame}{转化率与影响范围}
  \begin{itemize}
  \item {影响用户比例:IOS 21\%, ADR 15\%}
  \item {L-D: 三个环境一致,IOS 60\%, ADR 56\%}
  \item {
    D-O:
    \begin{itemize}
    \item {IOS: A 3.52\%, B 3.43\%, C 3.57\%}
    \item {ADR: A 2.92\%, B 2.96\%, C 2.96\%}
    \end{itemize}
  }
  \end{itemize}
\end{frame}

\section*{总结思考}

\begin{frame}
  \begin{enumerate}
  \item {
    特征精简做的不够
    \begin{itemize}
    \item {三个无区分度的特征}
    \item {进店铺属性与跟团游的相似度高}
    \end{itemize}
  }
  \item {
    试验需要继续:
    \begin{itemize}
    \item { 尝试新的特征 }
    \item { 调整个性化属性的加权量 }
    \end{itemize}
  }
  \end{enumerate}
\end{frame}

\end{CJK}

\end{document}
