% Copyright 2004 by Till Tantau <tantau@users.sourceforge.net>.
%
% In principle, this file can be redistributed and/or modified under
% the terms of the GNU Public License, version 2.
%
% However, this file is supposed to be a template to be modified
% for your own needs. For this reason, if you use this file as a
% template and not specifically distribute it as part of a another
% package/program, I grant the extra permission to freely copy and
% modify this file as you see fit and even to delete this copyright
% notice. 

\documentclass{beamer}
% Replace the \documentclass declaration above
% with the following two lines to typeset your 
% lecture notes as a handout:
%\documentclass{article}
\usepackage{CJKutf8}
\usepackage[T1]{fontenc}
%\usepackage[utf8x]{inputenc}
\usepackage{graphicx}


% There are many different themes available for Beamer. A comprehensive
% list with examples is given here:
% http://deic.uab.es/~iblanes/beamer_gallery/index_by_theme.html
% You can uncomment the themes below if you would like to use a different
% one:
%\usetheme{AnnArbor}
%\usetheme{Antibes}
%\usetheme{Bergen}
%\usetheme{Berkeley}
%\usetheme{Berlin}
%\usetheme{Boadilla}
%\usetheme{boxes}
%\usetheme{CambridgeUS}
%\usetheme{Copenhagen}
%\usetheme{Darmstadt}
%\usetheme{default}
%\usetheme{Frankfurt}
\usetheme{Goettingen}
%\usetheme{Hannover}
%\usetheme{Ilmenau}
%\usetheme{JuanLesPins}
%\usetheme{Luebeck}
%\usetheme{Madrid}
%\usetheme{Malmoe}
%\usetheme{Marburg}
%\usetheme{Montpellier}
%\usetheme{PaloAlto}
%\usetheme{Pittsburgh}
%\usetheme{Rochester}
%\usetheme{Singapore}
%\usetheme{Szeged}
%\usetheme{Warsaw}

\begin{document}
\begin{CJK}{UTF8}{gbsn}

\title{竞价排名系统-VPAS}

% A subtitle is optional and this may be deleted
\subtitle{功能简介、结构设计与接入标准}

\author{李庚\inst{1}}
% - Give the names in the same order as the appear in the paper.
% - Use the \inst{?} command only if the authors have different
%   affiliation.

\institute[Qunar.com] % (optional, but mostly needed)
{
  \inst{1}%
  旅游度假SI \\
  新业务部
}
% - Use the \inst command only if there are several affiliations.
% - Keep it simple, no one is interested in your street address.

\date{Jan. 20th, 2014}
% - Either use conference name or its abbreviation.
% - Not really informative to the audience, more for people (including
%   yourself) who are reading the slides online

\subject{2014技术分享}
% This is only inserted into the PDF information catalog. Can be left
% out. 

% If you have a file called "university-logo-filename.xxx", where xxx
% is a graphic format that can be processed by latex or pdflatex,
% resp., then you can add a logo as follows:

% \pgfdeclareimage[height=0.5cm]{university-logo}{university-logo-filename}
% \logo{\pgfuseimage{university-logo}}

% Delete this, if you do not want the table of contents to pop up at
% the beginning of each subsection:
\AtBeginSubsection[]
{
  \begin{frame}<beamer>{纲要}
    \tableofcontents[currentsection,currentsubsection]
  \end{frame}
}

% Let's get started

\begin{frame}
  \titlepage
\end{frame}

\begin{frame}{纲要}
  \tableofcontents
  % You might wish to add the option [pausesections]
\end{frame}

% Section and subsections will appear in the presentation overview
% and table of contents.
\section{功能介绍}

\begin{frame}{系统目标}
  \begin{itemize}
  \item {
    满足产品供应商通过提高点击单价获取展示的需求
  }
  \item {
    供应商之间可以通过互相竞争点击单价的方式为自己的产品博取更靠前的展示排名
  }
  \item {
    更灵活的点击计费模式:供应商自主控制参与竞价排名的产品和竞价预算
  }
  \end{itemize}
\end{frame}

\begin{frame}{应用举例}
  \begin{block}{度假搜索列表页}
    \includegraphics[scale=0.22]{./imgs/overview-qunar}
  \end{block}
\end{frame}

\begin{frame}{应用举例}
  \begin{block}{百度知心卡片}
    \includegraphics[scale=0.3]{./imgs/overview-baidu}
  \end{block}
\end{frame}

\subsection{供应商产品投放}
\begin{frame}{供应商产品投放}
  \begin{itemize}
  \item {
    自主选择自己的产品,加入竞价排名
    \pause
  }
  \item {
    设置投放计划(包括有效时间、日消耗限额等),实时操控竞价产品上下线
  }
  \end{itemize}
\end{frame}


\subsection{竞价展示排序}

\begin{frame}{竞价展示排序}
  \includegraphics[scale=0.25]{./imgs/overview-baidu}
  \begin{itemize}
  \item {
    接收竞价查询词参数,产生满足匹配条件的竞价产品排名
  }
  \item {
    综合产品质量以及点击单价计算排名
  }
  \end{itemize}
\end{frame}

\subsection{点击计费结算}

\begin{frame}{点击计费结算}
  \begin{itemize}
  \item {
    实时点击扣费
  }
  \item {
    无效点击识别,实时返款
  }
  \item {
    产生日结算报表:综合考虑一天的点击情况和客户的在线情况,产生用户的消费数据,然后和一天的实时结算数据比对,产生平帐数据,给用户返款或者扣款
  }
  \end{itemize}
\end{frame}

% \subsection{Second Subsection}

% You can reveal the parts of a slide one at a time
% with the \pause command:
%% \begin{frame}{Second Slide Title}
%%   \begin{itemize}
%%   \item {
%%     First item.
%%     \pause % The slide will pause after showing the first item
%%   }
%%   \item {   
%%     Second item.
%%   }
%%   % You can also specify when the content should appear
%%   % by using <n->:
%%   \item<3-> {
%%     Third item.
%%   }
%%   \item<4-> {
%%     Fourth item.
%%   }
%%   % or you can use the \uncover command to reveal general
%%   % content (not just \items):
%%   \item<5-> {
%%     Fifth item. \uncover<6->{Extra text in the fifth item.}
%%   }
%%   \end{itemize}
%% \end{frame}

\section{结构设计}

\subsection{技术基础}

\begin{frame}{技术基础}
\begin{block}{QADS文字链广告系统}
  \includegraphics[scale=0.23]{./imgs/qads-demo}
  \pause
\end{block}
\begin{block}{稳定的功能:主要的产品线上都有应用}
  \begin{itemize}
    \item {
      竞价展示排序 
    }
    \item {
      点击计费结算 
    }
  \end{itemize}
\end{block}
\end{frame}

\subsection{难点与挑战}

\begin{frame}{难点与挑战}
  \begin{block}{功能整合}
    \begin{itemize}
      \item {
        原本的度假业务系统在初始设计上无竞价的考虑
      }
      \item {
        原本的QADS文字链广告系统在设计上与业务产品无关
      }
    \end{itemize}
    \pause
  \end{block}
  \begin{block}{需求和技术的矛盾}
    \begin{itemize}
    \item {
      需求:希望尽快看到度假产品竞价接入的效果
    }
    \item {
      技术:实现通用的竞价接入框架,避免过度耦合
    }
    \end{itemize}
    \pause
  \end{block}
  \begin{block}{项目总开发时间较短}
    \begin{itemize}
      \item {
        设计上出错补救的机会几乎没有
      }
      \item {
        较难处理一开始不够清晰的产品需求
      }
    \end{itemize}
  \end{block}
\end{frame}

\subsection{系统设计原则}

\begin{frame}{系统设计原则}
  \begin{block}{严格定义系统边界}
    业务属性与竞价属性严格分离,互不处理对方的逻辑
    \pause
  \end{block}
  \begin{block}{避免对功能稳定的部分做修改或者重新开发}
    \begin{itemize}
      \item {
        关键模块逻辑复杂,且容易产生灾难性的结果。例如:扣费结算
      }
      \item {
        回归测试代价高,容易拖慢整体上线进度
      }
    \end{itemize}
    \pause
  \end{block}
  \begin{block}{简洁为美}
    \begin{itemize}
      \item {
        非必要的交互全部省略
      }
    \end{itemize}
  \end{block}
\end{frame}

\subsection{关键设计}

\begin{frame}{关键设计}
  \begin{block}{数据同步(Consistent)}
    \begin{itemize}
      \item {
        意义:保持竞价活力
      }
      \item {同步模型: Eventual Consistency}
      \item {避免Race Condition: 带时间戳的消息队列}
    \end{itemize}
  \end{block}
\end{frame}

\subsection{开发经验}

\begin{frame}{开发经验}{跨业务、跨团队的分布式系统}
  \begin{block}{开发、自测结束时的状况}
    \begin{itemize}
      \item {
        系统A: 接口A1,接口A2,...
      }
      \item {
        系统B: 接口B1,接口B2,...
      }
      \item {
        A, B各自通过联调的case
      }
    \end{itemize}
    \pause
  \end{block}

  \begin{block}{如果某组接口失效}
    \begin{itemize}
      \item {
        数据同步模型能否保证?
        \pause
      }
      \item {
        拔线测试:在一个正常运行的环境里,轮流令这些接口失效
      }
      \item {
        修复了至少2个设计上的缺陷
      }
    \end{itemize}
  \end{block}
\end{frame}

\section{产品接入标准}

\subsection{形成标准的动机}

\begin{frame}{形成标准的动机}
  \begin{block}{推动行业发展}
    \begin{itemize}
      \item {
        POSIX, HTTP, TCP/IP, ...
        \pause
      }
      \item {
        技术部json数据标准:'... 接口的规范统一,会减少对接口格式学习的成本,以及接口错误处理的bug问题。'
        \pause
      }
    \end{itemize}
  \end{block}
  \begin{block}{提供一个可复用的竞价基础平台}
    \begin{itemize}
    \item {对运维:提供相同的服务标准}
    \item {对开发、QA:提供开发测试接入的依据}
    \end{itemize}
  \end{block}
\end{frame}

\subsection{标准框架}

\begin{frame}{标准框架}{竞价系统与业务系统的交互规范定义}
  \begin{block}{竞价系统}
    \begin{itemize}
    \item {接收竞价产品输入的信息,形成能够支持产品竞价排名的数据}
    \item {根据查询词,计算竞价排名结果}
    \end{itemize}
    \pause
  \end{block}
  \begin{block}{业务系统}
    \begin{itemize}
    \item {
      实现产品接入所需的接口
      \pause
    }
    \item {
      依据标准定义的时序,通过消息与竞价系统完成竞价产品的状态同步
      \pause
    }
    \item {
      传递查询词参数,获取竞价排名和点击链接,按照自身的业务需要完成页面渲染
    }
    \end{itemize}
  \end{block}
\end{frame}

% Placing a * after \section means it will not show in the
% outline or table of contents.
\section*{总结}

\begin{frame}{总结}
  \begin{itemize}
    \item {
      合理复用:大大节约开发测试成本
    }
    \item {
      跨业务的分布式系统不能假设对方接口100\%可用;短暂的不可用不能破坏系统的基本属性
    }
    \item {
      形成标准供后来人参照,不断完善。
    }
  \end{itemize}
\end{frame}


% All of the following is optional and typically not needed. 
%% \appendix
%% \section<presentation>*{\appendixname}
%% \subsection<presentation>*{For Further Reading}

%% \begin{frame}[allowframebreaks]
%%   \frametitle<presentation>{For Further Reading}
    
%%   \begin{thebibliography}{10}
    
%%   \beamertemplatebookbibitems
%%   % Start with overview books.

%%   \bibitem{Author1990}
%%     A.~Author.
%%     \newblock {\em Handbook of Everything}.
%%     \newblock Some Press, 1990.
 
    
%%   \beamertemplatearticlebibitems
%%   % Followed by interesting articles. Keep the list short. 

%%   \bibitem{Someone2000}
%%     S.~Someone.
%%     \newblock On this and that.
%%     \newblock {\em Journal of This and That}, 2(1):50--100,
%%     2000.
%%   \end{thebibliography}
%% \end{frame}

\clearpage
\end{CJK}
\end{document}
