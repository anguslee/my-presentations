\documentclass{beamer}
% Replace the \documentclass declaration above
% with the following two lines to typeset your 
% lecture notes as a handout:
%\documentclass[handout]{beamer}
\usepackage{CJKutf8}
\usepackage[T1]{fontenc}
%\usepackage[utf8x]{inputenc}
\usepackage{graphicx}
\usepackage{subfigure}
\usepackage{mathtools}
\usepackage{ulem}
\usepackage{url}
\usepackage{pifont}
\usepackage{pgfplots}
\usepackage{textcomp}

\usepgfplotslibrary{external}
\pgfplotsset{width=10cm,compat=1.9}

%% \usepackage{tikz}
%% \usepackage{verbatim}
%% \usepackage[active,tightpage]{preview}
%% \PreviewEnvironment{center}
%% \setlength\PreviewBorder{10pt}

\usetikzlibrary{shapes,arrows}

\tikzexternalize
\everymath{\displaystyle}

% There are many different themes available for Beamer. A comprehensive
% list with examples is given here:
% http://deic.uab.es/~iblanes/beamer_gallery/index_by_theme.html
% You can uncomment the themes below if you would like to use a different
% one:
%\usetheme{AnnArbor}
%\usetheme{Antibes}
%\usetheme{Bergen}
%\usetheme{Berkeley}
%\usetheme{Berlin}
%\usetheme{Boadilla}
%\usetheme{boxes}
%\usetheme{CambridgeUS}
%\usetheme{Copenhagen}
%\usetheme{Darmstadt}
%\usetheme{default}
%\usetheme{Frankfurt}
%\usetheme{Goettingen}
%\usetheme{Hannover}
%\usetheme{Ilmenau}
%\usetheme{JuanLesPins}
%\usetheme{Luebeck}
%\usetheme{Madrid}
%\usetheme{Malmoe}
%\usetheme{Marburg}
%\usetheme{Montpellier}
%\usetheme{PaloAlto}
%\usetheme{Pittsburgh}
%\usetheme{Rochester}
%\usetheme{Singapore}
%\usetheme{Szeged}
\usetheme{Warsaw}

%% \newcommand{chartnumproducts} {
%% }



\begin{document}

\begin{CJK}{UTF8}{gbsn}

\title{2017.06晋级规则讲解}

% A subtitle is optional and this may be deleted
\subtitle{JAVA开发方向}

\author{李庚}
% - Give the names in the same order as the appear in the paper.
% - Use the \inst{?} command only if the authors have different
%   affiliation.

\institute[Qunar.com] % (optional, but mostly needed)
{
  \inst{1}
  TC \\
  Qunar.com
}
% - Use the \inst command only if there are several affiliations.
% - Keep it simple, no one is interested in your street address.

\date{\today}

\begin{frame}
  \titlepage
\end{frame}

\begin{frame}{职级体系的变化}
  原则:与老职级的评审规则兼容,不增加答辨评审的额外成本
  \begin{itemize}
  \item {
    新老级别转换后,对技术能力的要求大体不变,有微调。依然通过对各项能力打分的方式决定是否通过答辨考核。
  }
  \item {
    级别细分:P12, P13分别扩展成三个级别
    \begin{itemize}
      \item { 级别之间的差距缩小,晋级难度降低。}
    \end{itemize}
  }
  \item { Q16级开始,经过公司TC评审。Q15以下的晋级BU内部决定。}
  \item { 细则参考:\url{http://wiki.corp.qunar.com/pages/viewpage.action?pageId=155943526}}
  \end{itemize}

\end{frame}

\begin{frame}{打分标准调整}
  \begin{itemize}
  \item 高级别的晋升重视分享和培养新人的能力
  \item 具体的技术能力要求不变
  \item 相临的档位标准,分差由1变为2
  \end{itemize}
  
\end{frame}

\begin{frame}
  \begin{tabular}{l | c | c}
    \hline
    \textbf{开发能力} & \textbf{旧打分标准} & \textbf{新打分标准} \\
    \hline
        { 掌握所在部门要求的 \\
          开发工具与CM工具,\\
          代码符合公司开发命名规范,\\
          了解公共组件库使用规则 } & {1} & {1} \\
    \hline
            { 有良好的单元测试习惯,\\
              开发的单元测试代码能\\
              够覆盖主要逻辑分支,\\
              考虑到边界条件 } & {2} & {3} \\
    \hline
  \end{tabular}

\end{frame}

\begin{frame}{晋级资格说明}
  申请职级Q15至Q20(原13级及以上),需要满足技术培养积分条件
  \begin{itemize}
  \item {
    \emph{满10分}
  }
  \item {细则参考:\url{http://wiki.corp.qunar.com/pages/viewpage.action?pageId=63243061}}
  \end{itemize}
  
\end{frame}


\begin{frame}{如何准备}
  \begin{enumerate}
  \item {
    回顾对自己工作的理解
  }
  \item {
    阐述自己解决了哪些具体的技术问题,带来了什么结果
  }
  \item {
    技术细节展示:
    \begin{itemize}
    \item { 代码:书写风格,遵循的标准 }
    \item \emph{故障改进:方案与具体case}
    \end{itemize}
  }
  \item {
    控制好讲稿篇幅和演讲时间:25min陈述,5min提问
  }
  \end{enumerate}
\end{frame}

\begin{frame}{Q \& A}
  \begin{itemize}
  \item {
    个人职级查询:\url{http://ehr.corp.qunar.com}
  }
  \item {
    JAVA晋级标准:\url{http://wiki.corp.qunar.com/pages/viewpage.action?pageId=155943526}
  }
  \item {
    晋级资格积分要求:\url{http://wiki.corp.qunar.com/pages/viewpage.action?pageId=63243061}
  }
  \end{itemize}
\end{frame}


\end{CJK}

\end{document}
