\documentclass[handout]{beamer}
% Replace the \documentclass declaration above
% with the following two lines to typeset your 
% lecture notes as a handout:
%\documentclass[handout]{beamer}
\usepackage{CJKutf8}
\usepackage[T1]{fontenc}
%\usepackage[utf8x]{inputenc}
\usepackage{graphicx}
\usepackage{subfigure}
\usepackage{mathtools}
\usepackage{ulem}
\usepackage{url}
\usepackage{pifont}
\usepackage{pgfplots}
\usepackage{textcomp}

\usepgfplotslibrary{external}
\pgfplotsset{width=10cm,compat=1.9}

%% \usepackage{tikz}
%% \usepackage{verbatim}
%% \usepackage[active,tightpage]{preview}
%% \PreviewEnvironment{center}
%% \setlength\PreviewBorder{10pt}

\usetikzlibrary{shapes,arrows}

\tikzexternalize
\everymath{\displaystyle}

% There are many different themes available for Beamer. A comprehensive
% list with examples is given here:
% http://deic.uab.es/~iblanes/beamer_gallery/index_by_theme.html
% You can uncomment the themes below if you would like to use a different
% one:
%\usetheme{AnnArbor}
%\usetheme{Antibes}
%\usetheme{Bergen}
%\usetheme{Berkeley}
%\usetheme{Berlin}
%\usetheme{Boadilla}
%\usetheme{boxes}
%\usetheme{CambridgeUS}
%\usetheme{Copenhagen}
%\usetheme{Darmstadt}
%\usetheme{default}
%\usetheme{Frankfurt}
%\usetheme{Goettingen}
%\usetheme{Hannover}
%\usetheme{Ilmenau}
%\usetheme{JuanLesPins}
%\usetheme{Luebeck}
%\usetheme{Madrid}
%\usetheme{Malmoe}
%\usetheme{Marburg}
%\usetheme{Montpellier}
%\usetheme{PaloAlto}
%\usetheme{Pittsburgh}
%\usetheme{Rochester}
%\usetheme{Singapore}
%\usetheme{Szeged}
\usetheme{Warsaw}

%% \newcommand{chartnumproducts} {
%% }



\begin{document}

\begin{CJK}{UTF8}{gbsn}

\title{如何准备晋级答辨}

% A subtitle is optional and this may be deleted
%\subtitle{}

\author{李庚}
% - Give the names in the same order as the appear in the paper.
% - Use the \inst{?} command only if the authors have different
%   affiliation.

\institute[Qunar.com] % (optional, but mostly needed)
{
  \inst{1}
  Corp TC \\
  Qunar.com
}
% - Use the \inst command only if there are several affiliations.
% - Keep it simple, no one is interested in your street address.

\date{\today}

\begin{frame}
  \titlepage
\end{frame}

\begin{frame}{晋级考核的基本原则}
  考察\emph{工程师}对\emph{计算科学技术}本身的\emph{理解}以及\emph{熟练应用}的程度
  \begin{enumerate}
  \item <2-> {
    技术基础是否扎实?
  }
  \item <3-> {
    应用计算科学技术解决了什么问题?
    \begin{itemize}
    \item 业务方面
    \item 技术方面
    \end{itemize}
  }
  \item <4-> {
    工程应用是否熟练?
    \begin{itemize}
    \item 对需求是否正确理解
    \item 实现目标的资源代价是否合理
    \end{itemize}
  }
  \end{enumerate}
\end{frame}

\begin{frame}{能力考核模型}
  \begin{itemize}
  \item  {
    \emph{开发能力}:\uncover<2-> { 给定清晰的需求描述之后,正确实现需求目标的能力。}
  }
  \item  {
    \emph{运维能力}:\uncover<3-> { 利用现有的工具和服务,将系统正确运行并持久维护的能力。}
  }
  \item  {
    \emph{业务能力}:\uncover<4-> { 正确理解业务场景,并转化成技术层面合理设计方案的能力。}
  }
  \item  {
    \emph{架构能力}:\uncover<5-> { 对于复杂的需求场景,选择合理的技术方案和成熟的技术组件解决问题的能力。}
  }
  \item  {
    \emph{学习培养}:\uncover<6-> { 将工作中的经验积累总结,形成可供推广的技术实践方法的能力。 }
  }
  \end{itemize}
\end{frame}


\begin{frame}{答辨基本要求}
  \begin{enumerate}
  \item {
    阐述对自己工作内容的理解
  }
  \item {
    详述自己的工作解决了哪些具体的问题,带来了什么样的结果
    \begin{itemize}
    \item { 叙述思路:\emph{问题分析 \textrightarrow 技术解决方案 \textrightarrow 量化结果} }
    \item { 围绕前面的几个考核方面阐述,切忌流水账 }
    \end{itemize}
  }
  \item {
    细节秀:
    \begin{itemize}
    \item { 代码:书写风格,遵循的标准 }
    \item \emph{故障改进:方案与具体case}
    \end{itemize}
  }
  \item {
    控制好讲稿篇幅和演讲时间:25min陈述,5min提问
  }
  \end{enumerate}
\end{frame}

\begin{frame}{避免答辨的常见误区}
  \begin{enumerate}
  \item {
    “我具体做了什么”比“为什么要做”重要。
    \uncover <2-> {
      \begin{itemize}
      \item 无地放矢:被质疑业务能力
      \item 重新发明轮子:被质疑架构能力
      \end{itemize}
    }
  }
  \item {
    评委们都是大牛,自己讲的内容他们一听就懂。所以也不用太细致的讲。
    \uncover <3-> {
      \begin{itemize}
      \item { 评委不会来自你自己的技术团队,甚至很可能不会来自你经常合作的团队。}
      \item { 评委并不了解你的工作,答辨演讲的目标就是让评委了解你的能力。 }
      \end{itemize}
    }
  }
  \end{enumerate}
  
\end{frame}

\begin{frame}{Q \& A}
  \begin{itemize}
  \item {
    个人职级查询:\url{http://ehr.corp.qunar.com}
  }
  \item {
    晋级资格积分要求:\url{http://wiki.corp.qunar.com/pages/viewpage.action?pageId=63243061}
  }
  \end{itemize}
\end{frame}

\begin{frame}{晋级标准细则参考}
  \begin{itemize}
  \item { \emph{JAVA开发}: \url{http://wiki.corp.qunar.com/pages/viewpage.action?pageId=155943526} }
  \item { \emph{PHP \& PYTHON开发}: \url{http://wiki.corp.qunar.com/confluence/pages/viewpage.action?pageId=79242519} }
  \item { \emph{客户端开发}:\url{http://wiki.corp.qunar.com/confluence/pages/viewpage.action?pageId=79251833} }
  \item { \emph{前端开发}:\url{http://wiki.corp.qunar.com/confluence/pages/viewpage.action?pageId=78572334} }
  \item { \emph{QA}: \url{http://wiki.corp.qunar.com/confluence/pages/viewpage.action?pageId=63243228} }
  \item { \emph{DBA \& OPS}: \url{http://wiki.corp.qunar.com/confluence/pages/viewpage.action?pageId=78572475} }
  \end{itemize}
\end{frame}



\end{CJK}

\end{document}
